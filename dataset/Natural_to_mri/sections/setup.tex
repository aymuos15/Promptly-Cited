\section{Experiment Setup}

\subsection{Datasets}


We consider two tasks --- Brain age prediction and Alzheimer's disease (AD) detection from brain MRIs.
Our setup for brain age prediction is the same as~\cite{gupta2021improved}.
We used MRI scans from a subset of healthy subjects from the UK Biobank dataset~\cite{ukbb} with no psychiatric diagnosis.  The training, test, and validation set sizes were 7,312; 940; and 2,194; with a mean chronological age and standard deviation of 62.6 and 7.4 years.

We used the same dataset as~\cite{lam20203d} for the AD task. In particular, we use MRI scans from three phases of the  Alzheimer’s Disease Neuroimaging Initiative (ADNI), known as ADNI1,
ADNI2/GO, and ADNI3. These datasets have repeated scans of the same subject. However, the subjects in train/test/validation sets do not overlap. We used
4,561 scans (2,372/873/916 in train/validation/test).  The test set contains 244 scans with AD and 672 with a healthy diagnosis.

All the scans were reoriented to a standard brain MRI template, and the final dimensions of each scan in both datasets were $91\times109\times91$.

\subsection{Model \& Training}
We evaluate the 2D-Slice-CNNs under different settings --- with and without positional encodings and with and without pretrained encoders. We consider two encoder architectures.  We use a 5-layer 2D-CNN  encoder (similar to~\cite{gupta2021improved, lam2020accurate}) adapted from the 3D-CNN baseline of~\cite{peng2019accurate} to benchmark the effect of incorporating positional encodings.

We evaluate the effect of pretraining on natural images (2D) with pretrained ResNets~\cite{he2016deep}. We use ResNets pretrained on ImageNet-1K~\cite{deng2009imagenet} from the PyTorch model hub%
\footnote{\url{https://pytorch.org/hub/}}
as the 2D slice encoder. To get the slice embeddings from ResNets, we remove the final feed-forward layers and consider the output from the last layer (i.e., the average pooling layer). Therefore, the embedding sizes are 512 and 2048 for ResNet-18 and ResNet-50 encoders. For the 5-layer 2D-CNN encoder, the  embedding size is 32. When using pretrained weights, we finetune the entire model, including the encoder. Our implementation is available at \codeurl. 



For 2D-Slice  models, we find that slicing MRI along the sagittal axis works best for brain age prediction, as also observed by~\cite{gupta2021improved}. However, this was not the case for the AD task. Therefore we report the results of slicing along all three axes for AD.  We use the same hyperparameters as~\cite{gupta2021improved}, except for the optimizer and learning rate.
We search in \{(\adam, \four), (\sgd, \three / \four / \five)\} for the brain age prediction task.
\adam\ optimizer with a learning rate of \four\ works best for all the models except for the 3D-CNN. We report results for all models on the AD task with a learning rate of \four\ and \adam\ optimizer\footnote{We evaluated 3D-CNN with \sgd\ on the AD task, but \adam\ worked best.}. All the models are trained for 100 epochs, and the best model was chosen based on performance on the validation set at the end of every epoch.










\begin{table*}[t]
    \centering
    \fontsize{10}{11}\selectfont
    \begin{tabular}[]{l cccc ccc}
        \toprule
        \multicolumn{4}{c}{Method}
        &\multirow{2}{*}{\makecell[tc]{Bal.\\ Accuracy}}
        &\multirow{2}{*}{\makecell[tc]{F1\\Score}}
        &\multirow{2}{*}{\makecell[tc]{Avg.\\ Precision}}
        \\
        \cmidrule(l){1-4}
        Encoder &
        Pos. Enc.  &
        Pretrained &
        Axis &
        \\
        \cmidrule(l){1-1}
        \cmidrule(lr){2-2}
        \cmidrule(lr){3-3}
        \cmidrule(lr){4-4}
        \cmidrule(lr){5-5}
        \cmidrule(lr){6-6}
        \cmidrule(l){7-7}

        3D-CNN                & -    & \no  & -          & \result{88.40}{0.703} & \result{84.44}{1.028} & \result{92.33}{0.492} \\
        \cmidrule(l){1-7}

        2D-CNN (Mean)         & \no  & \no  & \axisone   & \result{87.53}{1.311} & \result{84.40}{1.707} & \textbf{\result{94.86}{0.946}} \\ %
        2D-CNN (Mean)         & \no  & \no  & \axistwo   & \result{85.78}{2.811} & \result{81.38}{3.610} & \result{90.57}{2.976} \\ %
        2D-CNN (Mean)         & \no  & \no  & \axisthree & \result{86.34}{1.445} & \result{82.21}{1.573} & \result{90.81}{0.677} \\ %
        \noalign{\vskip 0.25ex}\cdashline{2-7}\noalign{\vskip 0.5ex}
        2D-CNN (Mean)         & \yes & \no  & \axisone   & \result{87.05}{1.115} & \result{83.40}{0.873} & \result{93.40}{0.956} \\ %
        2D-CNN (Mean)         & \yes & \no  & \axistwo   & \result{85.41}{2.521} & \result{80.62}{2.521} & \result{89.86}{0.771} \\ %
        2D-CNN (Mean)         & \yes & \no  & \axisthree & \result{86.61}{0.942} & \result{81.20}{1.971} & \result{91.44}{2.123} \\ %
        \cmidrule(l){1-7}
        2D-ResNet-18 (Mean)   & \no  & \no  & \axisone   & \result{85.71}{4.834} & \result{80.89}{6.111} & \result{91.12}{2.880} \\ %
        2D-ResNet-18 (Mean)   & \no  & \no  & \axistwo   & \result{85.73}{1.433} & \result{81.30}{1.802} & \result{90.60}{2.544} \\ %
        2D-ResNet-18 (Mean)   & \no  & \no  & \axisthree & \result{84.61}{2.099} & \result{79.21}{2.906} & \result{88.95}{1.741}   \\ %
        \noalign{\vskip 0.25ex}\cdashline{2-7}\noalign{\vskip 0.5ex}
        2D-ResNet-18 (Mean)   & \yes & \no  & \axisone   & \result{86.57}{2.871} & \result{81.69}{3.972} & \result{91.80}{2.150} \\ %
        2D-ResNet-18 (Mean)   & \yes & \no  & \axistwo   & \result{84.34}{1.706} & \result{79.75}{2.258} & \result{90.29}{0.645} \\ %
        2D-ResNet-18 (Mean)   & \yes & \no  & \axisthree & \result{86.55}{1.059} & \result{81.39}{1.752} & \result{90.82}{0.701} \\ %
        \cmidrule(l){1-7}
        2D-ResNet-18 (Mean)   & \no  & \yes & \axisone   & \result{87.09}{1.192} & \result{82.07}{1.167} & \result{90.80}{1.601} \\ %
        2D-ResNet-18 (Mean)   & \no  & \yes & \axistwo   & \result{87.31}{0.971} & \result{83.19}{1.518} & \result{91.80}{1.626} \\ %
        2D-ResNet-18 (Mean)   & \no  & \yes & \axisthree & \textbf{\result{88.59}{1.187}} & \textbf{\result{85.10}{1.265}} & \result{93.11}{0.482} \\ %
        \noalign{\vskip 0.25ex}\cdashline{2-7}\noalign{\vskip 0.5ex}
        2D-ResNet-18 (Mean)   & \yes & \yes & \axisone   & \result{87.19}{2.037} & \result{83.16}{2.158} & \result{91.75}{1.291} \\ %
        2D-ResNet-18 (Mean)   & \yes & \yes & \axistwo   & \result{86.95}{2.750} & \result{82.33}{2.811} & \result{90.96}{1.541} \\ %
        2D-ResNet-18 (Mean)   & \yes & \yes & \axisthree & \textbf{\result{88.60}{2.058}} & \result{84.52}{2.389} & \result{92.56}{0.563} \\ %
        \bottomrule
    \end{tabular}
    \caption{Test set results for AD detection (binary labels) on the ADNI dataset. Higher is better for all metrics. Model selection was performed based on balanced accuracy. Results (mean and standard deviations) are reported over 5 runs with different seeds. The first column describes the neural network architecture or the encoder in the case of 2D-Slice models.}
    \label{tab:adni_results}
\end{table*}






\begin{table*}[t]
    \centering
    \fontsize{10}{11}\selectfont
    \begin{tabular}{l cc cc r}
        \toprule
        \multicolumn{3}{c}{Method}
        & \multirowcell{2}{Mean Absolute Err.\\ (MAE)}%
        & \multirowcell{2}{Root Mean Square Err.\\ (RMSE)}%
        & \multirow{2}{*}{\# Params}
        \\
        \cmidrule(l){1-3}
        Encoder &
        Pos. Enc. & Pretrained

        &
        &
        \\
        \cmidrule(l){1-1}
        \cmidrule(lr){2-2}
        \cmidrule(lr){3-3}
        \cmidrule(lr){4-4}
        \cmidrule(lr){5-5}
        \cmidrule(lr){6-6}

        3D-CNN                      & -    & \no  & \result{2.792}{0.032} & \result{3.521}{0.023} &  $\sim$ 3M  \\
        \cmidrule(lr){1-6}
        2D-CNN (Mean)               & \no  & \no  & \result{2.826}{0.021} & \result{3.582}{0.027} &  $\sim$ 1M  \\
        2D-CNN (Mean)               & \yes & \no  & \result{2.847}{0.051} & \result{3.617}{0.067} &  $\sim$ 1M  \\
        2D-CNN (Attention)          & \no  & \no  & \result{2.839}{0.009} & \result{3.590}{0.014} &  $\sim$ 1M  \\
        2D-CNN (Attention)          & \yes & \no  & \result{2.888}{0.067} & \result{3.655}{0.075} &  $\sim$ 1M  \\
        \cmidrule(lr){1-6}
        2D-ResNet-18 (Mean)         & \no  & \no  & \result{2.911}{0.039} & \result{3.684}{0.043} &  $\sim$ 12M \\
        2D-ResNet-18 (Mean)         & \no  & \yes & \textbf{\result{2.715}{0.029}} & \textbf{\result{3.426}{0.044}} &  $\sim$ 12M \\
        2D-ResNet-18 (Mean)         & \yes & \yes & \textbf{\result{2.721}{0.045}} & \textbf{\result{3.439}{0.053}} &  $\sim$ 12M \\
        \cmidrule(lr){1-6}
        2D-ResNet-50 (Mean)         & \no & \yes & \result{2.743}{0.018} & \result{3.468}{0.029} &  $\sim$ 26M \\

        \bottomrule
    \end{tabular}
    \caption{Test set results for the brain age prediction task. Lower MAE and RMSE are better. Model selection was performed based on MAE metric. The first column describes the neural network architecture or the encoder in the case of the 2D-Slice models. All experiments used \axisone\ slices.
    Results (mean and standard deviations) are reported over 5 runs with different seeds.}
    \label{tab:brain_age_results}
\end{table*}

